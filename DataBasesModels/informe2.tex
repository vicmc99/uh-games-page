\documentclass{article}
\usepackage[utf8]{inputenc}
\usepackage{geometry}
\usepackage{enumitem}
\usepackage{hyperref}
\geometry{a4paper, left=2.5cm, right=2.5cm, top=2cm, bottom=2cm}

\title{Modelo de Seguridad para la Base de Datos de la Aplicación Web}
\date{}
\begin{document}

\maketitle

\section*{Resumen:}
Este modelo de seguridad describe las medidas implementadas para garantizar la integridad, confidencialidad y disponibilidad de los datos en la base de datos de la aplicación web, siguiendo la arquitectura MVC. Se detallan las políticas de autenticación, autorización y protección de datos sensibles.

\section{Autenticación de Usuarios:}

\begin{enumerate}[label=1.\arabic*]
\item Usuarios se dividen en dos clases: Usuarios con privilegios y Usuarios sin privilegios.
\item Usuarios con privilegios (Administrador, Moderador, Periodista) deben proporcionar un nombre de usuario y una contraseña para autenticarse.
\item Las contraseñas se almacenan en la base de datos como hashes utilizando la función hash SHA-256.

Ejemplo de código en C# para autenticación:

\begin{verbatim}
public class Usuario
{
    public int Id { get; set; }
    public string NombreUsuario { get; set; }
    public string ContraseñaHash { get; set; }
    public string Rol { get; set; }
    // Otras propiedades
}

// Función para verificar las credenciales de un usuario
public bool VerificarCredenciales(string nombreUsuario, string contraseña)
{
    // Buscar al usuario en la base de datos por nombre de usuario
    var usuario = _dbContext.Usuarios.FirstOrDefault(u => u.NombreUsuario == nombreUsuario);

    if (usuario != null)
    {
        // Verificar la contraseña utilizando el hash almacenado
        var contraseñaHashIngresada = CalcularHashContraseña(contraseña, usuario.NombreUsuario);
        return usuario.ContraseñaHash == contraseñaHashIngresada;
    }
    return false;
}

// Función para calcular el hash de la contraseña
public string CalcularHashContraseña(string contraseña, string salt)
{
    using (var sha256 = SHA256.Create())
    {
        var contraseñaSalted = contraseña + salt;
        var hashBytes = sha256.ComputeHash(Encoding.UTF8.GetBytes(contraseñaSalted));
        return BitConverter.ToString(hashBytes).Replace("-", "").ToLower();
    }
}
\end{verbatim}

\end{enumerate}

\section{Autorización de Usuarios:}

\begin{enumerate}[label=2.\arabic*]
\item Se implementa un sistema de roles (Administrador, Moderador, Periodista) para controlar el acceso a las funcionalidades de la aplicación.
\item Se utilizan atributos de autorización en el código para restringir el acceso a ciertas funciones basadas en los roles de los usuarios.

Ejemplo de código en C# para autorización:

\begin{verbatim}
[Authorize(Roles = "Administrador")]
public ActionResult GestionBaseDatos()
{
    // Esta acción solo está disponible para usuarios con rol "Administrador".
    // Puedes realizar operaciones de gestión de base de datos aquí.
    return View();
}
\end{verbatim}

\end{enumerate}

\section{Tokens Temporales:}

\begin{enumerate}[label=3.\arabic*]
\item Se proporciona un token temporal a los usuarios con privilegios tras la autenticación.
\item El token tiene una duración de 2 minutos desde el último cambio realizado por el usuario.
\item Se usa una función hash que incluye el nombre de usuario, un número pseudoaleatorio y la fecha/hora de inicio de sesión para generar el token.

Ejemplo de código en C# para generación de tokens:

\begin{verbatim}
// Función para generar un token temporal
public string GenerarTokenTemporal(string nombreUsuario)
{
    // Generar el token utilizando una función hash
    var token = CalcularTokenTemporal(nombreUsuario);
    // Almacenar el token en la memoria temporal
    // y establecer su tiempo de expiración
    // ...
    return token;
}

// Función para calcular el token temporal
public string CalcularTokenTemporal(string nombreUsuario)
{
    // Calcular el hash del token
    // ...
    return tokenCalculado;
}
\end{verbatim}

\end{enumerate}

\section{Seguridad de Datos Sensibles:}

\begin{enumerate}[label=4.\arabic*]
\item Los datos sensibles se almacenan de forma segura en la base de datos, como contraseñas y otros datos confidenciales.
\item Las contraseñas se almacenan en forma de hash en la base de datos para evitar la exposición de contraseñas en texto claro.
\item Se limita el acceso a operaciones que involucran datos sensibles solo a roles autorizados (Administrador, Moderador).

Ejemplo de código en C# para seguridad de datos sensibles:

\begin{verbatim}
[HttpPost]
[Authorize(Roles = "Administrador, Moderador, Periodista")]
public ActionResult GuardarNoticia(Noticia noticia)
{
    // Guarda la noticia en la base de datos
    _dbContext.Noticias.Add(noticia);
    _dbContext.SaveChanges();
    return RedirectToAction("Index");
}
\end{verbatim}

\end{enumerate}

\section{Gestión de Vulnerabilidades:}

\begin{enumerate}[label=5.\arabic*]
\item Dado que el usuario final no tiene privilegios para realizar cambios en la aplicación, se minimiza la exposición a posibles vulnerabilidades.
\item Las credenciales de los usuarios privilegiados son supervisadas por el administrador, y se pueden cubrir en caso de compromiso.
\item Se recomienda realizar copias de seguridad de los datos importantes cada semana para evitar la pérdida de datos en caso de un ataque al administrador.
\end{enumerate}

\end{document}
